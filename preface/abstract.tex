
\abstract

Extracting high level features in data automatically can benefit various tasks such as classification and prediction. 
Deep belief networks allow to build feature extractors in an unsupervised manner. 
Utilizing a convolutional multilayered architecture can further improve the quality of the features on compositional data and allow a more abstract representation. 
Most architectures perform operations in discrete time steps and do not utilize a continuous information processing similar to the brain, which could lead to faster response times and a lower power consumption. 
In this thesis we propose different biologically inspired architectures to build and train a spiking deep belief network with a convolutional architecture. 
Two different training algorithms are presented.
The first approach trains the network in discrete time steps and the resulting network is afterwards transformed into a spiking neural network.
The second approach trains a spiking network directly with an adapted spike time dependent plasticity learning rule and weight synchronization.
Both algorithms are evaluated on different discrete and event-based datasets.
On the different datasets the algorithms are able to discriminate classes with high accuracy without any modification of the learning rules, thus indicating an adaptive feature extraction algorithm.
By comparing both approaches it becomes apparent that by introducing lateral inhibitory connections the directly trained algorithm is able to extract more discriminative features.
 
\chapter*{Zusammenfassung}

Die automatische Extraktion von "High-Level Features" unterstützt und erleichtert Aufgaben wie Klassifikation und Prädiktion.
Mit Hilfe von Deep Belief Networks können solche Feature Extraktoren unüberwacht gelernt werden.
Durch eine vielschichtige Convolutional Architektur kann die Feature Qualität auf kompositionellen Daten weiter verbessert werden und eine abstraktere Repräsentation ermöglicht werden.
Die meisten Architekturen arbeiten, im Gegensatz zum Gehirn, in diskreten Zeitschritten und nutzen keine kontinuierliche Informationsverarbeitung, was zu schnelleren Verarbeitungszeiten und einer niedrigeren Leistungsaufnahme führen könnte.
In dieser Arbeit werden verschiedene biologisch motivierte Architekturen vorgestellt, die es ermöglichen ein Convolutional Spiking Deep Belief Network aufzubauen und zu trainieren.
Es werden zwei verschiedene Ansätze präsentiert.
Der erste Ansatz trainiert ein Netzwerk in diskreten Zeitschritten und wandelt danach das trainierte Netzwerk in ein Spiking Netzwerk um.
Der zweite Ansatz trainiert direkt ein Spiking Netzwerk mit einer angepassten Version der "Spike-Time dependent plasticity" Lern-Regel mit Parameter-Synchronization.
Die Ansätze werden auf verschiedenen diskreten und Event-basierten Datensätzen evaluiert.
Beide Ansätze erreichen auf den verschiedenen Datensätzen eine hohe Klassifikationsgenauigkeit ohne die zugrunde liegenden Lern-Regel zu verändern, was auf einen adaptionsfähig Feature-Extraktions Algorithmus schließen lässt.
Der Vergleich der beiden Ansätze legt nahe, dass aufgrund von lateral-hemmenden Verbindungen der zweite Ansatz diskriminativere Features extrahieren kann. 

        