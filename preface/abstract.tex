
\abstract

Extracting high level features in data automatically can benefit various tasks such as classification and prediction. 
Deep belief networks allow to build feature extractors in an unsupervised manner. 
Utilizing a convolutional multilayered architecture can further improve the quality of the features on compositional data and allow a more abstract representation. 
Most architecture perform operations in discrete time steps and do not utilize a continuous information processing similar to the brain, which could lead to faster response times and a lower power consumption. 
In this thesis we propose different biologically inspired architectures to build and train a spiking deep belief network with a convolutional architecture. 
Two different training algorithms are presented.
One trains the network in discrete time steps and the resulting network is afterwards transformed into a spiking neural network.
Another approach trains a spiking network directly with an adapted spike time dependent plasticity learning rule and weight synchronization.
Both algorithms are evaluated on different discrete and event-based datasets.
On the different datasets the algorithms are able to discriminate classes with high accuracy without any modification of the learning rules, thus indicating a generalizable feature extraction algorithm.
By comparing both approaches it becomes apparent that by introducing lateral inhibitory connections the directly trained algorithm is able to extract more discriminative features.
 
\chapter*{Abstrakt}