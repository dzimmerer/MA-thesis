\chapter{Implementation}


\section{Analog DBNs}

The DBNs were implemented in the Theano-Framework, to utilize the computational power of the GPU.
The implementation of the singe RBMs is adapted from the official Theano RBM (REF).
\\
\\
An upward pass in performed as follows:
As an input we take a 4D tensor $\textbf{V}$ in the form of \textit{(batch size, input channels, input height, input width)}, and for the upwards pass convolve it with a filters $\textbf{W}$ of the shape \textit{(number of filters, input channel, filter height, filter width)} with a stride of 1.
This results in feature maps $\textbf{H}_{pre} = \textbf{W} * \textbf{V}$ of the shape \textit{(batch size, number of filters, input height - filter height + 1 , input width - filter width + 1)}.
A constant $c$ is added to the pre-activation feature maps as bias.
On those feature maps a sigmoid function $\sigma$ is applied and accordingly Bernoulli sampled to get the activation of the hidden units $\textbf{H}$.

\begin{algorithm}
\caption{Upward pass}
\begin{algorithmic}
\State ${H}_{pre}\gets W * V + c $  
\State ${H}_{sigmoid} \gets \sigma({H}_{pre})$
\State $H \gets {H}_{sigmoid} > Random()$
\end{algorithmic}
\end{algorithm}


The downward pass on the activation of the hidden units is performed similar:
The input is the activation of the hidden unit $\textbf{H}$, and convolve it the the 180 degree flipped kernel $\tilde{\textbf{W}}$, where the first and second dimensions are swapped.
Afterwards a sigmoid function $\sigma$ is applied and accordingly sampled to get the new visible layer activations $\textbf{V}'$.

\begin{algorithm}
\caption{Downward pass}
\begin{algorithmic}
\State ${V}_{pre}\gets \tilde{W} * H + c $  
\State ${V}_{sigmoid} \gets \sigma({V}_{pre})$
\State $V \gets {V}_{sigmoid} > Random()$
\end{algorithmic}
\end{algorithm}

One complete Gibbs sampling step consists of on upwards and a consecutive downward pass.

\begin{algorithm}
\caption{Gibbs step}
\begin{algorithmic}
\State $V_0 \gets upward(H_0) $  
\State $H_1 \gets downward(V_0) $  
\end{algorithmic}
\end{algorithm}

Each RBM is trained with the CD-$k$ update rule, therefore after performing one Gibbs sampling step to get the distribution of the data $\textbf{V}_0, \textbf{H}_0$  (positive phase), we perform $k-1$ additional Gibbs sampling steps to get an estimate of the model distribution $\textbf{V}_k, \textbf{H}_k$  (negative phase).

As error function we define the difference between the free energy of the positive phase and the negative phase.
\[
\Delta \textbf{W} = \frac{\partial F(V_k)}{\partial \textbf{W}} -  \frac{\partial F(V_0)}{\partial \textbf{W}} 
\]
We use Theanos auto-differentiation to determine the gradient and perform gradient decent with a learning rate $\mu$ and a weight decay of $\upsilon$.
\[
\textbf{W}' = \textbf{W} - \mu \Delta \textbf{W} - \upsilon \textbf{W} 
\]
We train the RBM for a given number of epochs, with a batch size smaller than the dataset, thus performing stochastic gradient descent.

To build up the DBN each RBM was trained greedily.
After training one RBM the entire dataset was converted by doing one upwards pass in the trained RBM and using the activation of the hidden units as the new input data for the next RBM.

\begin{algorithm}
\caption{build DBN}
\begin{algorithmic}
\For{$i$  in $\#$ \textit{layers}} 
\State Train $\text{RBM}_i$ on dataset $\text{ds}_i$
\State convert dataset: $\text{ds}_{i+1} = upward_i(\text{ds}_i)$ 
\EndFor
\end{algorithmic}
\end{algorithm}

Up to this time no label information is used to train any RBM, thus the learned features up to this point are trained purely unsupervised.
The last layer, the classification layer, consists of a fully connected RBM trained on the input data of the converted data set concatenated with a one-hot encoding of the label.



\section{Conversion}

To simulate the converted DBNs we use the pyNN framework with Nest as spiking network simulator.
To simulate the CNN and DBN we use current and conductance based LIF neurons, respectively.
Each unit is injected with high frequency Possion distributed excitatory and inhibitory input spikes with the frequency $\lambda_{noise}= 5000 \text{Hz}$ and synaptic weights $w_{n+}$ and $w_{n-}$ respectively.
%The parameters for the models can be seen in Table x( @TODO table).

A unit in the DBN is represented by a neuron, a layer by a neuron population.
For the connections static synapses with the scaled original weights were used. 

The input is transformed to a Poisson distributed spike trains, where the rate of the Poisson process $\lambda_{data}$ is proportional to the original data value (e.g. the image intensities). 
The input is directly fed into the bottom layer neurons with a high fixed weight $w_{in}$ to reliably generate equal spikes in the bottom layer.    

For each data sample the network is simulated for a runtime of $t$. 
To get the classification results, the spikes count of all single neurons in the label layer $a_i$ is recorded and to get a label prediction the index of the most frequent spiking neuron is determined:
\[
y_{pred} = argmax_i \; a_i
\]


\section{eCD}

The eCD learning was implemented in PyNN with Nest with simulated STDP weight updated at discrete time steps outside the simulation, as well as Brian simulator with event-based online STDP weight updates, but due to the simulation speed we chose Brian for most of our experiments.

As neuron type LIF neurons with high frequency input noise were chosen.

Each input element $x_i \in \textbf{x}$ of a data sample $\textbf{x}$ gets transformed to Poisson distributed spikes train, with the rate $\lambda$ proportional to the input value $\lambda \propto x_i$. 

Each RBM consist of a visible and hidden layer. 
The visible consist of one neuron population with $n = |\textbf{x}|$ neurons representing the input, the data population. 
If needed a second neuron population representing the label input $\textbf{y}$, the label population, can be added to the visible units.
The hidden layer is an neuron population of the size  $ k = \textit{number of filters} \times (\textit{input height} - \textit{filter height} + 1) \times (\textit{input width} - \textit{filter width} + 1)$.

The data population is sparsely connected to the hidden layer with synchronized weights implementing the convolution, while the label population is fully connected to the hidden layer.

In the hidden layer we connect a neuron to other neurons in a square around same position in different feature maps with a inhibitory synapse with a fixed negative weight.

The training of the spiking RBM is performed with the adapted eCD algorithm (see 4.3).
The RBM is trained on one data sample for $t_{sample} = n \times t_{ref}$ cycles, which can, considering the different training phases, be further divided into $t_{sample} = t_{burn-in} + t_{learn} + t_{burn-in} + t_{learn} + t_{flush} = t_{pos} + t_{neg} + t_{flush}$ (with $t_{pos} = t_{neg} = t_{burn-in} + t_{learn}$ ). 
As a result we receive the following training procedure:
\begin{itemize}

\item $t \in [0, t_{burn-in}]$ : In the first phase ("Data burn-in phase"), the visible layer is induced with a strong negative current and the data input is fed as spikes with a high synaptic weight, so that the visible layer only spikes in accordance to the input data and is unaffected from any spikes in the top layer.
The STDP learning flag is set to $g=0$, so no learning is allowed

\item $t \in (t_{burn-in} , \; t_{burn-in} + t_{learn}]$ : In the second phase ("Data distribution phase"), now the STDP learning flag is set to $g=1$ so positive learning is allowed.
This should drive the weights to represent the data distribution.

\item $t \in (t_{pos}, \;  t_{pos} + t_{burn-in}]$ : In the third phase ("Model burn-in phase"), we set the data input of the visible layer to zero and remove the induced negative current, to let it reach the model distribution.
In this phase the learning is disabled, setting the STDP learning flag to $g=0$

\item $t \in (t_{pos} + t_{burn-in}, \;  t_{pos} + t_{burn-in} + t_{learn}]$ : In the fourth phase ("Model distribution phase"), the STDP learning flag is set to $g=-1$, enabling only negative (un-)learning.
This will unlearn the model distribution.

\item $t \in (t_{pos} + t_{neg}, \;  t_{pos} + t_{neg} + t_{flush}]$ : In the optional fifth phase ("Flush phase"), we induce a strong inhibitory current to the visible and hidden layer to remove all activity and allow a fresh start in the first phase.

\end{itemize}


The weight synchronization is set at discrete time points, after $n$ training samples.
This is performed by taking the mean over all the weights, which are to be the shared.
In addition a small weight decay is introduced:
\[
W_{new} = \text{mean}(\textbf{W}_{group}) - \upsilon * \text{mean}(\textbf{W}_{group}) , 
\]
where $\upsilon$ is the weight decay rate and $\textbf{W}_{group} = (W_0 ,... W_n)$ are all the updated weights of synapses belonging to a group of synapses with the same shared weights.
\\
Each RBM is trained on $m$ data samples.
\\
After a RBM is trained, the next RBM will be trained on top of the previous RBM.
Therefore the a connection with a strong synaptic weight is established from the hidden layer of the previous RBM to the new RBM.
The original input data is still fed to the bottom RBM while the activations of hidden layer in previous RBM act as training data for the top RBM.


At test time the network is run for a fixed timespan $t_{test}$ with the test data fed to the bottom RBM.
There is no external input forwarded the label layer while the number of spikes in the label are counted.
The neuron with the most spikes represents the predicted label.  

 